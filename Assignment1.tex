%%%%%%%%%%%%%%%%% DO NOT CHANGE HERE %%%%%%%%%%%%%%%%%%%% {
\documentclass[12pt,letterpaper]{article}
\usepackage{fullpage}
\usepackage[top=2cm, bottom=4.5cm, left=2.5cm, right=2.5cm]{geometry}
\usepackage{amsmath,amsthm,amsfonts,amssymb,amscd}
\usepackage{lastpage}
\usepackage{enumerate}
\usepackage{fancyhdr}
\usepackage{mathrsfs}
\usepackage{xcolor}
\usepackage{graphicx}
\usepackage{listings}
\usepackage{hyperref}

\hypersetup{%
  colorlinks=true,
  linkcolor=blue,
  linkbordercolor={0 0 1}
}

\setlength{\parindent}{0.0in}
\setlength{\parskip}{0.05in}
%%%%%%%%%%%%%%%%%%%%%%%%%%%%%%%%%%%%%%%%%%%%%%%%%%%%%%%%%% }

%%%%%%%%%%%%%%%%%%%%%%%% CHANGE HERE %%%%%%%%%%%%%%%%%%%% {
\newcommand\course{Physics II}
\newcommand\semester{Fall 2019}
\newcommand\hwnumber{1}                 % <-- ASSIGNMENT #
\newcommand\NetIDa{Paul Smith}           % <-- YOUR NAME
\newcommand\NetIDb{}           % <-- STUDENT ID #
%%%%%%%%%%%%%%%%%%%%%%%%%%%%%%%%%%%%%%%%%%%%%%%%%%%%%%%%%% }

%%%%%%%%%%%%%%%%% DO NOT CHANGE HERE %%%%%%%%%%%%%%%%%%%% {
\pagestyle{fancyplain}
\headheight 35pt
\lhead{\NetIDa}
\lhead{\NetIDa\\\NetIDb}                 
\chead{\textbf{\Large Assignment \hwnumber}}
\rhead{\course \\ \semester}
\lfoot{}
\cfoot{}
\rfoot{\small\thepage}
\headsep 1.5em
%%%%%%%%%%%%%%%%%%%%%%%%%%%%%%%%%%%%%%%%%%%%%%%%%%%%%%%%%% }

\begin{document}

\section*{Problem 1}

You attempt a 105c (two-and-a-half flips forward in a tuck position) off of a 10 m diving platform. Assuming you can rotate at a maximum rate of one flip per second and you can jump a maximum of 1 m upwards off the board: a) what is the fastest initial velocity (upwards) that you can have? b) how much time passes before you hit the water? and c) can you complete the skill?

\section*{Problem 2}

A car accelerates at a rate of $25 \frac{\mathrm{km}}{\mathrm{hr*s}}$. How much time does it take to get to $125$ km/hr? How much distance will the car cover in this time?

\section*{Problem 3}

You decide to try and measure the speed of a car you're riding in by timing the distance between power poles. Your data is given in Table \ref{tab:q3}:

\begin{table}[h]    % [h] means to print the table here
    \centering  % to center the table https://www.overleaf.com/project/5d757e7e591aa30001b65c17
    \begin{tabular}{c|c} % one 'c' for each column. It means centered. You can use 'l' or 'r' for left and right, respectively. '|' prints a line

     Distance (m) & Time (s) \\ \hline
        0   &  0  \\
        80   & 2.5   \\
        168   &  5.1  \\
        245   &  7.6  \\
        315   &  9.1  \\

    \end{tabular}
\caption{Measurements from the back seat}
\label{tab:q3}
\end{table}

What is the overall average velocity? How about the average velocity of each measurement? Draw both the overall average and segment averages on a v-t graph.

\section*{Problem 4}

A train accelerates at a constant rate $a_c$ from rest. Draw a-t, v-t, and x-t graphs for the train. What is the overall shape of the position graph?

\end{document}